% **************************** Define Graphics Path **************************
\ifpdf
    \graphicspath{{Chapter4/Figs/Raster/}{Chapter4/Figs/PDF/}{Chapter4/Figs/}}
\else
    \graphicspath{{Chapter4/Figs/Vector/}{Chapter4/Figs/}}
\fi

\chapter{LIWC/text transformation chapter}\label{chap:liwc}

One of the key issues in machine learning for content moderation is that such systems both in deployed settings (see \autoref{chap:filter}) and in research (see \autoref{chap:intro} and \autoref{chap:nlp}) over-fit to individual tokens that see over-representation in the positive and negative classes respectively. While research efforts have been made to address such issues \cite{CITE: cite papers that try to address overfitting}, the problem of over-fitting to words and identity markers remain an open question for the field. While some such approaches have addressed this problem by replacing certain words and phrases with more general tokens \cite{CITE: Replacing token papers} or masking \cite{CITE: Masking token paper} tokens. Other work has attempted to address the problem by treating it as a problem of dataset bias \cite{Dixon:2018}. Here, I propose a different approach which serves to address the issue of models over-fitting to tokens by 1) minimising the vocabulary in order to avoid over-fitting to distributional skews of low-frequency tokens across classes; 2) representing documents in terms of how they represent thoughts, feelings, and personality; and 3) through such vocabulary minimisation highlight the importance of how words are used rather than their surface forms while retaining model performance. An additional benefit of such vocabulary reduction is a proportional reduction in model size and training time for complex models such as neural networks, resulting in models that have a smaller environmental impact \citep{Strubell:2019}.

Through the use of the Linguistic Inquiry and Word Count (LIWC) dictionary \cite{LIWC:2015,Original LIWC Citation}, I pre-process documents from large vocabularies, that are riddled with obfuscations and intentionally and unintentionally misspelled words into a smaller vocabulary set representing instead psycholinguistic properties of words. Through a reduction of thousands, or in some cases hundreds of thousands, of unique tokens to hundreds of LIWC categories, I aim for models to gain deeper insight into language patterns of abuse than simply selecting the most frequently used tokens. Moreover, I show that such a reduction is accompanied by a negligible intra and inter dataset performance in comparison to models using the full surface-token vocabularies.

Through the use of simple deep neural networks and `shallow' linear models, I show that through reducing the vocabulary sizes by up to $99\%$ and the number of model parameters by up to $99\%$, while increasing the depth of the information in the remaining vocabulary, it is possible to achieve comparable performances within datasets and mild improvements on out-of-domain datasets. This holds two strong implications for future research on computational hate speech detection: first that current approaches through an over-reliance on surface forms are computationally inefficient, and second that the exclusive use of surface forms of tokens can lead models to overly attend to the occurrence of certain tokens and variations (e.g. prominent misspellings) \citep{Rottger:2021}. Finally, as datasets for hate speech detection frequently contain biases along racialised and dialectal lines \citep{Waseem:2018,Davidson:2019}, the use of LIWC can serve as small aid in avoiding such biases as dialectal spellings of words are likely to not appear in the dictionary, thus being relegated to unknown tokens (see \autoref{tab:liwc_tok} for synthetic examples of LIWC representations). Thus, this chapter seeks to provide answers to the following research questions:

\begin{minipage}{0.9\textwidth}
\vspace{5mm}
    \begin{enumerate}[start=1, label={\textbf{RQ \arabic*}}]
        \item{\textit{Can LIWC provide a meaningful substitute to using words or sub-word tokens as input tokens and how is model performance affected by such a substitution?}}
        \item{\textit{What are the implications of using LIWC as input on model development and model size?}}
        \item{\textit{What are the implications on generalisability of LIWC-based models?}}
    \end{enumerate}
\end{minipage}

\newpage
\section{Previous work}

In the interest of curtailing the spread of online abuse, a large number of technical approaches have been considered in the ever-increasing body of research on the topic (please see \autoref{sec:nlp} for a broad overview on the topic). Here, I focus on three different strands of research. First, I briefly introduce the LIWC dictionary. Second, I consider manual development of features for machine learning models as it is necessary to form hypotheses for what might may serve as indicators of abuse on the basis of the dataset and problem in question. Third, I examine neural network approaches for abusive language detection. Finally, I consider the growing body of research devoted to examining the generalisability of computational models for abusive language detection. I restrict my attention to studies in conducted on abuse in English as it is most pertinent to this work.

\subsection{Linguistic Inquiry and Word Count}

The Linguistic Inquire and Word Count dictionary and software was initially developed by \citet{Pennebaker:2001} in an effort to address the issue of high disagreement and negative effects on well-being of judges, as they reviewed essays written on people's experiences of emotional upheaval. In order to minimise such costs, \citet{Pennebaker:2001} turned to computationally counting words that were in $80$ ``psychology-relevant categories'' in order to gain an understanding of the emotional states and cognition of the authors at the time of writing. By passing over a large body of text within a single document, e.g. personal essays, the \citet{Pennebaker:2001} compute how percentage occurrence of each invoked category. While there are some examples that appear clear cut, e.g. the categorisation of articles such as `a' and `the',
% TODO review footnote
\footnote{Though the categorisation of word classes may seem trivial, however which class a word is categorised into depends on the linguistic theory that a given classification is based on \cite{CITE: Wait for Adina}.}
other word classes, such as ``emotion word categories'' are more clearly subjective and require deeper human consideration \citep{Tauscik:2010}.
Though LIWC was initially developed using long form texts, the version of the dictionary that I use in this dissertation is an expanded version that also used Twitter and ``blogs'' in the development of the dictionary \citep{Pennebaker:2015}. As such, though not originally intended for the use on short-form messages, LIWC has evolved with the rise of new forms of communication in efforts for the dictionary to accurately reflect language use in short-form documents.\vspace{5mm}

In this thesis, I utilise LIWC to provide the word categories that each word invokes, and rather than compute the overall word classes exhibited in a document, I use the LIWC categories of each word in a document as an alternative representation of the document from which percentages of word categories invoked can be recovered. Thus, my approach diverges slightly in the goals of using LIWC, however it does not diverge in the method for obtaining information about the psychological state of the author.
In the interest of curtailing the spread of online abuse, a large number of technical approaches have been considered in the ever-increasing body of research on the topic (please see \autoref{sec:nlp} for a broad overview on the topic). Here, I focus on three different strands of research. First, I consider manual development of features for machine learning models as it is necessary to form hypotheses for what might may serve as indicators of abuse on the basis of the dataset and problem in question. Second, I examine neural network approaches for abusive language detection. Finally, I consider the growing body of research devoted to examining the generalisability of computational models for abusive language detection. I restrict my attention to studies in conducted on abuse in English as it is most pertinent to this work.

\subsection{Modelling}
\subsubsection{Manually Crafted Features}

A large body of work has sought to use manually developed features for online abuse detection \citep{Davidson:2017,Waseem:2017,Ibrohim:2019,Vega:2019,Wiegand:2018,Tian:2020,Kumar:2019,Fortuna:2018}, showing performance boosts from using manually developed features such as the predicted author gender \citep{Waseem-Hovy:2016} or Part-of-Speech (POS) tags \citep{Davidson:2017}. The primary reasons for using manually crafted features is two-fold: First, by using manually crafted features it is necessary to have some understanding of the data at hand and some intuition about which features may distinguish the classes from the data from one another. Second, as manual features are frequently used with models that don't use neural architecture, they allow for interpretable machine learning models, in the sense that one can often identify how each token contributed towards a final prediction. Moreover, as features are often computationally fast to compute, the use of features along with their expressive interpretability, allow for quickly testing hypothesis surrounding online abuse and its nature.
Considering a handful of systems that use some of the most frequently used features for the development of automated systems for detecting various forms of online abuse, distinct modelling choices, features and rationales for their use become prominent. Here I provide a brief overview of prominent features; how they are used, including which models and feature weighting schemes they are used with; and the explicit and implicit rationales for the use of each feature.

First, the most common feature used, and rarely used on its own, is a Bag-of-Words (BoW) \citep{Fortuna:2018,Davidson:2017}, where each token in a document is treated as independent from the remainder of the document. The use of this feature frequently relies on the use of stop-word lists to remove tokens that are bound to occur frequently across a majority of documents, such as determiners, to avoid models from learning spurious correlations with such words and an individual class due to fluctuations in the data. The understanding of abuse that underlies this feature, is that the occurrence of some tokens are likely to disproportionately occur in abusive contexts, and that those tokens, in isolation, will indicate abuse. Several works have complicated this notion \citep[e.g.]{Waseem:2018,Davidson:2019}, arguing that tokens, in isolation, do not provide the necessary context to determine whether a text is abusive and due to certain perspectives on abuse being overly represented \citep{Waseem:2016} in annotation guidelines and annotations, some words that have been reclaimed, and thus have an innocuous usage potentially in addition to an abusive use, may be disproportionately represented in the positive classes.

To address the issue of token independence, several approaches use n-grams, often bi-grams \citep{Waseem:2016} and tri-grams \citep{Davidson:2017} to aid with identifying abuse. Here, by considering groups of sequential token occurrences independently from one another, a step is taken away from the independence of individual tokens, instead to the independence of short sequences of tokens. Due to this remaining independence assumption, similar issues arise to the limitations of BoW hold for n-grams.

POS tags have also seen frequent use in abusive language detection tasks \citep{Fortuna:2018} and are often used as n-grams. The intuition behind the use of POS tags for abuse detection is that abuse may differ from non-abuse in terms of linguistic structure. While n-grams of POS tags with an independence assumption may not reveal the full depth of the linguistic syntax available through POS tagged data (in contrast to the POS tags of the entire sequence being treated as a single feature), it does relay \textit{some} information on the linguistic structure which has been proven helpful for predicting abuse \citep{Fortuna:2018}.

Another frequently used feature is sentiment analysis \citep{Fortuna:2018} with the underlying assumption that abuse and negative sentiment are correlated, and can thus aid in detecting some forms of abuse. Similarly to BoW and n-grams, this is a feature that is most frequently used in combination with other features as sentiment alone is not presumed to be a good predictor of abuse \citep{Fortuna:2018}. Sentiment as a feature, like the use of LIWC proposed in this dissertation, assumes that some higher level reasoning about the data can be helpful to automatically detecting abuse. Specifically, its use suggests that there the concepts of negativity and hostility towards entities will be relevant to detecting abuse in texts. Notably, some previous work that uses sentiment as a feature for abuse detection \citep{Davidson:2017} relies on previously built systems for detecting sentiment. An implication of using previously trained systems for computing sentiment, rather than assuming that sentiment can be extrapolated only from the dataset, is that sentiment and abuse, while correlated are not equated and thus that the task of detecting sentiment, while related is a distinct task from detecting abuse. As such, a sentiment and abuse are tasks that in some cases co-constitute each other while there may be no correlation in other cases.

Finally, LIWC has previously been proposed as a feature for the classification of abuse in a small number of studies \citep{Nina-Alcocer:2019,Joksimovic:2019}. In these studies, LIWC has been used in conjunction with other features such as lexical features (e.g. word n-grams) and syntactic features (e.g. POS tags) \citep{Joksimovic:2019}. This use of LIWC, similar to the motivations for its use in this chapter, relies on an assumption that the mental states of the speaker and the interpretations of readers will relay information on the intention of the speaker to cause offence. For instance, \citet{Nina-Alcocer:2019} compute the percentages of emotions that are expressed in abusive documents in efforts to identify correlations between impassioned speech and abuse, asserting an intuition that abusive speech is likely to occur in individual moments dominated by emotion rather than rationality. A position that \citep{Waseem:2016} argue is likely as they find that considering the top $100$ most frequently occurring tokens, ranked using Term Frequency - Inverse Document Frequency (TF-IDF), does not aid in the prediction of hate speech, suggesting that in many cases it may be a question of moments of abuse rather than consistently abusive people.

All features must be weighted, either through raw counts or their relative frequency. One such frequently used weighting scheme is TF-IDF which weights features by their relative frequency in the corpus \citep{Fortuna:2018}, assigning higher weight to the features that are rare corpus-wide and lower weights to those that common. As such, TF-IDF can be a useful measure to address the dominance of high-frequency tokens. At the same time, TF-IDF also increases the capacity for models to overfit to the corpus and generalise poorly, as tokens that are unique to a corpus may not exist in other data or even be common to other data. The use of n-grams as features provides a similar double-edged benefit, where models learn sequences of words, in abuse detection the most common n-grams are unigrams, bi-grams, and trigrams. Such word-sequences can be helpful for models in uncovering patterns of language use in the corpus but are also sensitive to the vocabulary changes that occur across datasets. For instance \citet{Waseem-Hovy:2016} train a Logistic Regression classifier and identify that character n-grams of innocuous words such as `Islam' and `Muslim' rank as some of the most predictive features due to the disproportionate occurrences of such terms in the hateful classes.

Many of the previously mentioned works use similar machine learning models, with a particular dominance of Logistic Regression and Support Vector Machines (SVMs) (please see \autoref{chap:nlp} for more detail). One notable exception to this is the work of \citet{Gorrell:2018}. In this work, the authors use a ``set of NLP tools, combining them into a semantic pipeline'' \citep[pp. 601]{Gorrell:2018}. Rather than using supervised classification techniques, a rule-based systems was developed to detect abuse that they argue allows for a interpretable and easy to modify method to address weaknesses of the approach without the need for additional large quantities of data.\footnote{This detail on the rule-based nature of the classification systems was provided by Genevieve Gorrell in personal communications.} However, this approach is a laborious one as it requires the researchers to manually identify patterns of abuse and construct rules that can address such patterns along with any exceptions to the patterns that are not abusive.
\vspace{5mm}

In this chapter, we take inspiration from the use of manually crafted features as a way to provide testable hypothesis while departing from the notion of feature generation. Specifically, we hypothesise that LIWC categories can provide deep information for predictive modelling that can allow for high performance in spite of token sparsity when using neural network methods.

\subsubsection{Neural Networks}\label{sec:liwc_nn}
Though the earliest models for the tasks were predominately linear models that used manually generated features \citep{Waseem-Hovy:2016,Davidson:2017,Warner:2012} more recent work has been dominated by the development of neural network based models for automated abuse detection, posting ever-evolving State-of-the-Art models and classification performances \citep[e.g.]{Park:2017,Badjatiya:2017,Zimmerman:2018,Stoop:2019,Isaksen:2020}. Here I consider a handful of neural network methods for detecting abuse, focusing on the distinct implications following the modelling choices and the logics that underpin them. As all neural network based methods that I examine receive only the text as input, the primary differences between the models is in their use and organisation of different types of layers and the loss function selected for the respective models.

The most commonly used architecture for neural networks that in the surveyed literature is a CNN \citep{Park:2017,Gamback:2017,Wulczyn:2017,Kolhatkar:2021,Zimmerman:2018,Wang:2020}. As CNNs have been the subject of particularly interest, a number of distinct modelling approaches have been proposed. First, relying a simple neural network architecture, \citet{Kolhatkar:2021} use GloVe embeddings as the first layer, followed by three convolutional layers (that have window sizes $3, 4,$ and $5$, respectively) with global maximum pooling layers. Prior to passing to an output layer, dropout is applied to the output of the convolutional layers which is then passed to a dense layer. For all layers prior to the output layer use a ReLU (see \autoref{chap:nlp} for more detail) activation function. The output layer applies the sigmoid function to provide a prediction from the model. This model most closely resembles the CNN architecture used in this chapter. As this model uses a pre-trained word-embedding layer as its input layer, the input the model receives are documents that have been subject to tokenisation processes.

A different architecture is proposed by \citet{Park:2017}. In their work they compare a single classifier, what they dub a `one-step classifier', that predicts the final classes directly with a stacked architecture of two models, or a `two-step classifier' in their vernacular that first predicts whether content is abusive and second predicts which type of abuse the documents predicted as abusive are. There are two governing distinctions between the two-step architecture proposed by \citet{Park:2017} and the architecture proposed by \citet{Kolhatkar:2021}. First, \citet{Kolhatkar:2021} acts as a one-step classifier whereas the architecture proposed by \citet{Park:2017} acts in two steps. Second, \citet{Kolhatkar:2021} only acts on documents tokenised into words and punctuation whereas \citet{Park:2017} propose a CNN that takes documents tokenised into words and punctuation in addition to documents tokenised into characters. \citet{Park:2017} show that through the use of a one-step CNN trained on word and character input, they achieve a performance boost obtaining a F1-score of $0.827$ on the datasets proposed by \citet{Waseem-Hovy:2016} and \citet{Waseem:2016}, though the performance boost is lost once a two-step hybrid CNN is used.

As CNNs build feature mappings by passing over the data using filters, they come with certain assumptions built into them. As researchers define the number of filters and the stride size, they also define the range within which they believe that relevant terms are likely to occur. The implication of this is then that there will likely be some, potentially overlapping, ranges that the models learn patterns from. Depending on how researchers define these, the models will develop feature mappings corresponding to the ranges provided.

Another frequently used architecture is LSTMs \citep{Badjatiya:2017,Kolhatkar:2021,Meyer:2019}. Here, \citet{Kolhatkar:2021} propose using a bi-directional LSTM that, like their CNN, has a pre-trained embedding layer, a recurrent layer, a dropout layer, and a fully connected output layer with a sigmoid activation to predict the output classes. \citet{Meyer:2019} on the other hand take develop on the idea of a hybrid CNN, developing a LSTM architecture that takes documents tokenised into words and characters as input. The word representation is obtained through tokenisation passed through an embedding layer and the character representation is obtained by processing the documents with a CNN.
Using these approaches, \citet{Kolhatkar:2021} show comparable performances between the CNN and bi-directional LSTM on their dataset. \citet{Meyer:2019} on the other hand show that a baseline model only using character level information performs comparably with other more complex approaches, obtaining a macro F1-score of $0.7923$ for the baseline and $0.7924$ for the final system on the dataset proposed by \citet{Waseem-Hovy:2016}, and notably outperforms several other previously proposed methods.

The use of LSTMs, that rely on recurrence, break with the independence assumption of the manual feature-based models. By recurring over a document, each new token is considered in conjunction with the previous tokens that have not been forgotten. In this way, an assumption is built into the models that through processing enough token sequences, it will be possible to identify patterns that connote abuse. Such a reliance on the text alone does not consider the positionality of abuse; \citet{Waseem:2018} argue only through understanding the context within which the speaker and audience exist in, is it possible to deem something as abusive. For instance, it is only through an understanding of the speaker that one can deem whether the \textit{n-word} is weaponised as abuse or is reclaimed to connote complex social identity.\vspace{5mm}

All methods described that rely on documents tokenised into sentences rely on pre-trained embedding layers (most frequently GloVe \citet{Pennington:2014}) that come with their own benefits and costs. For instance, word embeddings that are trained on web-text are likely to harbour social biases \citep{Bolukbasi:2016} that have been proven hard to address \citep{Gonen:2019}. On the other hand, they also allow for better representations of related concepts and will be less susceptible to creating different representations for closely related concepts as a result of dataset biases. For instance, the concepts `Television' and `T.V.' might only be distantly related, if at all, in a small dataset due to few co-occurrences within the dataset. In a larger dataset, spanning millions of documents, these two concepts are likely to appear as closely related as a robust language representation will likely have been achieved for such commonly occurring tokens.
The methods that rely on character embeddings are also subject to similar distributional concerns, however this can be a benefit when used in conjunction with word embeddings. As there are only a much smaller set of possible characters compared to words, less data is needed to train robust embedding layers, though the trained character embeddings will be particularly attuned to the dataset at hand. On the other hand, due to such particularity of the character embeddings, they are less likely to map well onto other domains even if they show good performance on the dataset that they are derived from.\vspace{5mm}

For the work in this chapter, the use of pre-trained embeddings is not appropriate for some models. Specifically the models that use LIWC-represented documents as LIWC embeddings are not publicly available or have been developed, to the best of my knowledge. Moreover, documents represented through LIWC categories are poorly suited for training general embeddings as only a small set of tokens are defined and they are not necessarily distributed in a fashion suitable for developing generalised such embeddings. Second, I don't use pre-trained embeddings in the architectures for all other models to ensure that any comparisons with the LIWC-based models are a direct comparison of the influence of using LIWC as input tokens and avoiding potentially confounding factors.




\subsection{Datasets}
Beyond the different specific models and types of input layer, the datasets themselves also differ strongly from one another across a number of attributes: the various sizes of the datasets, the platforms the datasets have been sampled from, the annotator selection, the annotation guidelines, and the aims of the datasets. As we seek to build generalisable models and compare the performance across datasets, we reduce the classification task in each dataset to a binary `abuse/not-abuse' class, as each dataset addresses different aspects and notions of abuse.
We train our models on either the Twitter dataset collected by \citet{Davidson:2017} and the Wikipedia Talk pages dataset sampled by \citet{Wulczyn:2016}, respectively, as our training datasets. We choose these datasets in part due to their sizes and in part due to the different breadth of communicative styles. To estimate cross-dataset performance, we evaluate our models on \citet{Waseem-Hovy:2016}, \citet{Waseem:2016}, and \citet{Garcia:2019}.

\zw{INSERT: Table of BPE and LIWC vocabularies}
\zw{Table of token differences and intersections between classes}
\zw{INSERT: Short paragraph on vocabularies.}

\subsubsection{Training Datasets}
Our first dataset for training is the dataset developed by \citet{Davidson:2017} consists of $24,784$ tweets that are sampled from Twitter using keywords obtained from \citet{Hatebase}. The dataset is annotated for ``hate speech'', ``offensive language'' and ``neither''. The collection rationale was that not all content that immediately appears to be abusive is necessarily that, and that hate speech models must be able to distinguish between what is offensive and what is hateful \cite{Davidson:2017} (please see \autoref{chap:nlp} and \autoref{chap:filter} for more in-depth discussions on the implications of label categories, their overlap and differences). The dataset was annotated by crowd-workers on FigureEight\footnote{Previously known as CrowdFlower}. Unlike most datasets for abuse, this dataset consists primarily of positive instances, with $77$\% of the (binarised) dataset belonging to the positive class.

Our second dataset used for training is the dataset presented by \citet{Wulczyn:2016}. This dataset consists of more then $100,000$ comments from Wikipedia talk pages that have been annotated for personal attacks and toxicity \cite{Wulczyn:2016}. The rationale of this dataset is that personal attacks are harmful to ongoing conversations, and that through the identification and removals of comments that poison, or toxify online conversations, more space will be left for healthy and constructive discussions (please see \autoref{chap:filter} for an in-depth consideration of the politics of what constitutes ``toxic'' and ``healthy''). The binarised distribution of documents tagged for toxicity aligns better with prior research, with the positive class consuming $\approx 9$\% of the dataset.

\subsubsection{Evaluation Datasets}
For our evaluation datasets, we use \citet{Waseem-Hovy:2016}, a dataset of $16,000$ documents that are sampled from Twitter and annotated for ``racism'', ``sexism'', and ``neither''. The dataset was annotated by two coders, who labelled $31$\% of the dataset containing as either ``sexist'' or ``racist'' content. This dataset was developed for an early exploration into automated content moderation of online hate speech. We also use the dataset by \citet{Waseem:2016} that followed this first exploration. Here the annotation guidelines remain the same while the label-set is expanded to include the intersection of racism and sexism, the ``both'' category. This dataset contains $6000$ documents, labelled by intersectional feminist activists, and another label-set annotated by crowd-workers from FigureEight. We choose the intersectional feminist tagged annotations, as \citet{Waseem:2016} show that simple computational models perform better using this tagset. The binarised positive labels occupy $15.19$\% of the dataset. Finally, we use the dataset on white-supremacist speech developed by \citet{Garcia:2019}. This dataset, unlike the previous two evaluation datasets does not stem from Twitter, but instead the data is collected from StormFront\footnote{www.stormfront.net}, a web forum dedicated to the preservation and dissemination of white supremacist ideology. This dataset contains \zw{INSERT: Number of total document counts when using the entire dataset instead of the balanced one.} documents labelled for being hateful or not hateful, with \zw{INSERT: Percentage of positive class docs} in the positive class.

\subsubsection{Dataset and Platform Affordances}

As the datasets differ quite significantly in the sizes of the raw number of documents as well as the vocabulary sizes. Moreover, as the datasets are selected from different websites with different communities, purposes, and means of interaction; the data sampled from each platform may differ in content as well as style. Considering for instance \citet{Waseem:2016}, this dataset was collected on Twitter while the maximum length of a tweet was $140$ characters. Documents are thus short as they are given an upper limit on the number of characters. On the other hand, \citet{Wulczyn:2017} collected their data from the Wikipedia Editor Talk pages, where comments are not limited by length. Additionally, these two domains differ in that conversations on Twitter may have no particular topic, conversations on Wikipedia Talk pages always refer back to a specific topic and the conversation of how to address a particular edit to a page. Finally, given Wikipedia's ongoing issues with recruiting editors from a diverse set of backgrounds \cite{CITE: Wikipedia editors issue} and Twitter's comparatively broad user base \cite{CITE: Twitter userbase by demographic ref} may influence which dialects are represented on the platforms, which patterns of speech (e.g. sociolects, slang, and shorthand) occur, and the style of the discussions and conversations.

\subsubsection{Annotator Selection}

There are some interesting discrepancies in the selection of annotators for the datasets that we apply our models to. \citet{Waseem:2016} select their annotators based on socio-political positions, controlling for a specific interpretation of abuse. On the other hand \citet{Wulczyn:2017} select their annotators from the users of the Wikipedia Talk pages. However, the Wikipedia editor community has been accused of being a highly male space that is unwelcoming to women \cite{CITE: Cite article talking about anti-women culture on wikipedia}. This suggests that the influence of their selection of annotators, who are culturally situated in the norms and culture of the Wikipedia editor community, are also likely to be less attuned to content that may be offensive to women, but is accepted communicative practices within the Wikipedia editor community.

On the other hand \citet{Garcia:2019} and \citet{Davidson:2017} select annotators that are removed from the context of the documents they are annotating. This suggests that global understandings of what constitutes abuse are possible, and that it is possible to annotate without a deep understanding of the issues and communicative practices of the specific communities that are being investigated.

While we accept that the influence of annotation guidelines have strong influences on the subject that is being examined (e.g. \citet{Davidson:2017} examine the differences in what is merely offensive and what is hateful, \citet{Garcia:2019} examine what is hateful from a white supremacist community and what is not, and \citet{Wulczyn:2017} examine things that make conversations toxic and hostile), we assume that these different guidelines and questions highlight different aspects of abuse. Through our efforts to develop methods that can identify different forms of abuse across different datasets, we accept the assumption that there are some global understandings of abuse that can be learned by machine learning models. We revisit this assumption in \autoref{chap:disembodied}.

\section{Modelling}
As our aim is to understand how representing documents using only the tokens derived using the Linguistic Inquiry and Word Count dictionary may influence the generalisability of models for hate speech detection across datasets and domains, we use simple modelling architectures and feature representations.

\subsection{Datasets}

\subsection{Pre-processing}


\begin{table}
  \centering
  \resizebox{\textwidth}{!}{%
  \begin{tabular}{l|l|l}
    Document                   & Word Token Representation & Byte-Pair Representation\\\hline
    Man I fucking hate animals & Man I fucking hate animals & \_man \_i \_fucking \_hate \_animals \\
    Man I fking h8 animals     & Man I fking h8 animals     & \_man \_i \_f king \_h 0 \_animals   \\
    Bruv I fking hate animals  & Bruv I fking hate animals  & \_br uv \_i \_f king \_hate \_animals
  \end{tabular}%
  }
  \caption{Word token and BPE representation.}
  \label{tab:bpe_tok}
\end{table}

\begin{table}[]
\centering
\footnotesize
\begin{tabular}{l|p{10.5cm}}
Document                   & LIWC Representation \\ \hline
Man I fucking hate animals & MALE\_SOCIAL PPRON\_FUNCTION\_I\_PRONOUN AFFECT\_SEXUAL\_BIO\_INFORMAL\_NEGEMO\_ANGER\_ADJ\_SWEAR AFFECT\_NEGEMO\_ANGER\_VERB\_FOCUSPRESENT UNK \\\hline
Man I fking h8 animals     & MALE\_SOCIAL PPRON\_FUNCTION\_I\_PRONOUN UNK NUM UNK \\\hline
Bruv I fking hate animals  & UNK PPRON\_FUNCTION\_I\_PRONOUN UNK AFFECT\_NEGEMO\_ANGER\_VERB\_FOCUSPRESENT UNK
\end{tabular}
\caption{Examples of LIWC representations.}
\label{tab:liwc_tok}
\end{table}


\subsection{Neural Models}\ref{sec:redux_neural}

\zw{Add this where it fits}
Given that onehot encoded models are much larger in size, I limit the exploration of hyper parameter values to embedding dimension, hidden dimension, dropout and learning rate. Particularly dropout and learning rate are of interest as onehot encoded models are larger than index encoded models. For this reason, I explore deviate from the best parameters identified by the index encoded models. For embedding and hidden dimensions on the other hand, I follow the best parameters found by the index encoded models.

We implement and train four different model architectures with two variations each, resulting in a total of 8 different neural models for each input representation. We choose to train a Multi-Layered Perceptron, a Recurrent Neural Network, a Long-Short Term Memory network, and a Convolutional Neural Network. We select these four models as they have each been used in previous work \cite{CITE: Find papers with Neural approaches for each of the models}. Each of the four models are trained with either a linear input layer and an embedding input layer. We choose to make this distinction as our datasets are too small to meaningfully learn embedding layers, yet for the LIWC transformed documents, pre-trained embeddings have little value, as all tokens would be out-of-vocabulary. Thus to compare comparable entities, we train a model with each type of input layer. This difference in the input layers however also implies a difference in how each document is represented: for the models with linear input layers, each document is encoded as a onehot tensor whereas the models that use an embedding layer as it's input layer the documents are represented as index-encoded tensors (please see \autoref{fig:onehot_embedding} for an illustration). We implement all neural network models using PyTorch \cite{CITE: Pytorch paper} and use gradient clipping to prevent the issue of exploding gradients \cite{Bengio:1994}.
In order to focus on the utility of the transformed document representations, we use bare bones models with simple architectures. To address the issue of the models over-fitting to the data either by identifying spurious correlations or by over-training the model, we subject each model to dropout and early stopping criteria (please see \autoref{sec:dropoutearly} for more details on the functionality of early stopping and dropout).

\begin{figure}
  \centering
  \includegraphics[scale=0.75]{onehot_embedding.jpg}
  \caption{Onehot and Index encoded tensors.}
  \label{fig:onehot_embedding}
\end{figure}

\subsection{Hyper-Parameter Search}

We perform Bayesian Hyper Parameter Tuning using Weights and Biases \cite{Wandb} to identify the optimal parameters for each model and the variants of each model, evaluated using macro-F1 score. We also investigate the influence of batch sizes and learning rate. For more in depth explanation of how each model works, please refer to \autoref{sec:model_background}.\vspace{5mm}
For each embedding-based model, we sample $200$ different hyper-parameter settings for each of our models. Due to the number of hyper-parameter combinations that we sample, we find multiple competing hyper-parameter settings that achieve comparative results. Here we only consider the top 
% TODO: INSERT NUMBER OF PARAMETER SETTINGS TO SEARCH
the hyper-parameter hyper-parameter settings and report the setting that performs best on the training data and the test data. In an initial, aborted, hyper-parameter search of the onehot encoded models, we identified that the hyper-parameter search with $200$ different parameter settings would take more than 200 days to complete.
To limit the search space, we search only the top
% TODO: INSERT NUMBER OF PAREMETERS TO SEARCH
hyper-parameter settings from the hyper-parameter search of the embedding-layer models to search. See \autoref{tab:appendix-embedding-hyperparam} and \autoref{tab:appendix-onehot-hyperparam} in \autoref{Appendix:hyperparams} for full listing of hyper-parameter values.

\begin{landscape}
\begin{table}[]
\centering
\begin{tabular}{l|llll|llll}
                      & \multicolumn{4}{c|}{BPE}                 & \multicolumn{4}{c}{LIWC} \\
                       & MLP     & CNN      & RNN     & LSTM    & MLP     & CNN     & RNN     & LSTM     \\ \hline
Embedding Dimension    &         &          &         &         &         &         &         &          \\
Hidden Dimension       &         &          &         &         &         &         &         &          \\
Window Size            &         &          &         &         &         &         &         &          \\
Batch Size             &         &          &         &         &         &         &         &          \\
Learning Rate          &         &          &         &         &         &         &         &          \\
Dropout                &         &          &         &         &         &         &         &          \\
Activation Function    &         &          &         &         &         &         &         &          \\
\# Epochs               &         &          &         &         &         &         &         &          \\
Validation F1-score    &         &          &         &         &         &         &         &
\end{tabular}
\caption{Best hyper-parameter setting for neural models with embedding input layer trained on \citet{Davidson:2017}.}
\label{tab:redux_hyperparam_search_davidson}
\end{table}
\end{landscape}


\begin{landscape}
\begin{table}[]
\centering
\begin{tabular}{l|llll|llll}
                      & \multicolumn{4}{c|}{BPE}                 & \multicolumn{4}{c}{LIWC} \\
                      & MLP     & CNN     & RNN     & LSTM    & MLP     & CNN     & RNN     & LSTM    \\ \hline
Embedding Dimension   &         &         &         &         &         &         &         &         \\
Hidden Dimension      &         &         &         &         &         &         &         &         \\
Window Size           &         &         &         &         &         &         &         &         \\
Batch Size            &         &         &         &         &         &         &         &         \\
Learning Rate         &         &         &         &         &         &         &         &         \\
Dropout               &         &         &         &         &         &         &         &         \\
Activation Function   &         &         &         &         &         &         &         &         \\
\# Epochs              &         &         &         &         &         &         &         &         \\
Validation F1-score   &         &         &         &         &         &         &         &
\end{tabular}
\caption{Best hyper-parameter setting for neural models with embedding input layer trained on \citet{Wulczyn:2017}.}
\label{tab:redux_hyperparam_search_wulczyn}
\end{table}
\end{landscape}

For all models trained, we perform hyper-parameter tuning over the same batch size, learning rate, and the maximum number of training epochs. The values for the learning rate are sampled from a uniform distribution while the batch size and epoch count are sampled from a categorical distribution. More generally, the values for all hyper-parameters, asides from dropout and the learning rate, are sampled from a categorical distribution. The trial values for Dropout and the learning rate are both sampled from a uniform distribution. We select a Rectified Linear Unit \cite{CITE: RELU paper} activation function for all models except for the Recurrent Neural Network and the Long-Short Term Memory model, as the latter is only implemented with a $Tanh$ activation function in PyTorch and we use the former to ensure comparison between the Long-Short Term Memory model and all other models.

\begin{itemize}
  \item Maximum epoch count: $\{50, 100, 200\}$,
  \item Batch size: $\{16, 32, 64\}$,
  \item learning rate: $[0.00001, 1.0]$
\end{itemize}

\subsubsection{Multi-Layered Perceptron}

The first neural model that we use, is a Multi-Layered Perceptron. We choose the model as it is a simple neural network, that can act as a minimal setting of the usefulness of neural networks. Our Multi-Layered Perceptron consists of either a linear input layer or an embedding input layer. The obtained representation of a batch of documents are then passed on to a linear hidden layer and passed on to a linear output layer, which is subject to a softmax layer computing probability estimates for each class. Following the first and second layer of the model architecture, we subject the output of the layer to a non-linear activation function and a dropout layer. The model architecture is depicted in \autoref{fig:liwc_mlp}. For the Multi-Layered Perceptron models, we search over the following values:

\begin{itemize}
  \item dropout probability: $[0.0, 0.5]$,
  \item hidden layer dimension: $\{64, 100, 200, 300\}$, and
  \item the activation function: $\{ReLU\}$
\end{itemize}

For the models that use an embedding layer as their input layers, we additionally search for the dimension of the embedding layer, allowing the model to search between $[100, 300]$. For the linear input layer model, the hidden dimension search functionally replaces the search over the embedding size.

\begin{figure}
  \centering
  \includegraphics[scale=0.75]{mlp.jpg}
  \caption{Multi-Layered Perceptron model architecture.}
  \label{fig:liwc_mlp}
\end{figure}


\subsubsection{Recurrent Neural Network}

The second neural model we implement a Recurrent Neural Network; we choose this model as it offers improvements over Multi-Layered Perceptron due to the introduction of the recurrence over the tokens in the documents (see \autoref{chap:nlp} for more detail). Our Recurrent Neural Network consists of an input layer, which can be a linear layer or an embedding layer, a recurrent neural network layer, a linear output layer, a dropout layer, and a softmax layer to compute the probabilities of each class. The recurrent neural network layer is provided an activation function, which is applied within the layer.

The model is trained by first passing batches of index or onehot encoded documents through the input layer, and are passed on to the recurrent neural network layer.\footnote{We use the PyTorch implementation of the Recurrent Neural Network layer.} The resulting representation is then subject to a dropout layer before it subject to a linear layer that maps to the number of output classes. Finally, the softmax layer computes the probability estimates for each class. See \autoref{fig:liwc_rnn} for a depiction of the models. We set the activation function for the recurrent neural network to $\tanh$.

For the Recurrent Neural Networks, we perform a hyper-parameter tuning over the following parameters and values:

\begin{itemize}
  \item dropout probability: $[0.0, 0.5]$,
  \item embedding layer dimension: $\{64, 100, 200, 300\}$,
  \item hidden layer dimension: $\{64, 100, 200, 300\}$, and
   \item the activation function: $\{Tanh, ReLU\}$
\end{itemize}

\begin{figure}
  \centering
  \includegraphics[scale=0.75]{rnn.jpg}
  \caption{Recurrent Neural Network model architecture.}
  \label{fig:liwc_rnn}
\end{figure}

\subsubsection{Long-Short Term Memory}

The Long-Short Term Memory network that we implement, consists of an input layer, that similarly to the RNN and MLP can be either a linear layer or an embedding layer; a one-directional Long-Short Term Memory network layer;\footnote{We use the PyTorch implementation of the Long-Short Term Memory Network layer.} an output layer; a dropout layer; and a softmax layer to compute the probabilities. The implementation of the Long-Short Term Memory layer is such that it always uses \textit{Tanh} as its non-linear activation function. We use Long-Short Term Memory networks due to their prior successes in other works \cite{CITE: LSTM papers} and because they present a development over RNNs, in that they identify information to ``forget'' in to address the issue of long-range dependencies that occur (please see \autoref{chap:nlp} for more detail).

The model is trained by passing batches of documents through the input layer prior to feeding them into the Long-Short Term Memory network layer. The output of the Long-Short Term Memory network layer is then subject to the dropout layer, before the output layer maps down to the number of label classes. Finally, the softmax layer is used to obtain an estimation of the probability distributions for each class (please see \autoref{fig:liwc_lstm} for depiction of model architecture.).

For these models, our hyper-parameter tuning considers the following parameters and values:

\begin{itemize}
  \item dropout probability: $[0.0, 0.5]$,
  \item embedding layer dimension: $\{64, 100, 200, 300\}$, and
  \item hidden layer dimension: $\{64, 100, 200, 300\}$
\end{itemize}

\begin{figure}
  \centering
  \includegraphics[scale=0.75]{lstm.jpg}
  \caption{Long-Short Term Memory Network model architecture.}
  \label{fig:liwc_lstm}
\end{figure}

\subsubsection{Convolutional Neural Network}

For our final neural model type, we use a Convolutional Neural Network. We select this model as it has been applied previously in academic research \cite{CITE: CNN papers} and in industry (e.g. the Perspective API\footnote{https://github.com/conversationai/perspectiveapi}). Similarly to the previous model types, the input layer of the Convolutional Neural Network models can either be an embedding layer or a linear layer. The second layer of the model is a two-dimensional convolutional layer. Finally, there is an output layer and a softmax layer (See \autoref{fig:liwc_cnn} for depiction of model architecture).

For these models, we only consider the activation function, embedding, and hidden dimension in our hyper-parameter tuning, in addition to batch size and learning rate.

\begin{itemize}
  \item window size: $\{(1, 2, 3), (2, 3, 4), (3, 4, 5)\}$,
  \item Number of filters: $\{64, 128, 256\}$,
  \item hidden layer dimension: $\{64, 100, 200, 300\}$, and
  \item the activation function: $\{ReLU\}$
\end{itemize}

\begin{figure}
  \centering
  \includegraphics[scale=0.75]{cnn.jpg}
  \caption{Convolutional Neural Network model architecture.}
  \label{fig:liwc_cnn}
\end{figure}

\subsection{Baseline Models}

We develop several different baseline methods to compare our method with. For each shallow baseline model (i.e. Logistic Regression and Support Vector Machines), we train two different types: a surface-token based model that uses the surface forms of the documents (e.g. words), and a LIWC based model. For each of these models, we represent each document for training and classification as a bag-of-words after removing stop words. In addition to the aforementioned models, we also train deep neural networks that similarly rely on surface forms. Specifically, we use the models described in \autoref{sec:redux_neural} providing surface level tokens as the input to the models.

\subsubsection{Baseline hyper-parameters}

Similarly to our neural models, we perform a parameter search to identify the optimal parameters for training our linear baseline models. For the Support Vector Machine models, we explore a regularisation strength of $C \in \{0.1, 0.02 \ldots 1.0\}$ and $penalty \in \{L1, L2\}$. For the Logistic Regression models, we explore the same values of $C$ and append \texttt{elasticnet} to the possible space of penalties, yielding $penalty \in \{L1, L2, elasticnet\}$. We report the optimal settings in \autoref{tab:liwc_baseline_linear_params}.

\zw{EDIT: Update this table based on binary results}

\begin{table}[]
\centering
\resizebox{\textwidth}{!}{%
\begin{tabular}{l|cccc|cccc}
                      & \multicolumn{4}{c|}{BPE}                                                             & \multicolumn{4}{c}{LIWC}                                                             \\ \hline
                      & \multicolumn{2}{c}{Logistic Regression} & \multicolumn{2}{c}{Support Vector Machine} & \multicolumn{2}{c}{Logistic Regression} & \multicolumn{2}{c}{Support Vector Machine} \\ \hline
                      & C               & Penalty               & C                 & Penalty                & C               & Penalty               & C                 & Penalty                \\ \hline
\cite{Davidson:2017}  & $1.0$           & L2                    & $0.1$             & L2                     & $0.4$           & L2                    & $0.1$             & L2                     \\
\cite{Wulczyn:2016}   & $1.0$           & L2                    & $0.2$             & L2                     & $1.0$           & L2                    & $1.0$             & L2
\end{tabular}%
}
\caption{Optimal parameters for linear Support Vector Machine baselines.}
\label{tab:liwc_baseline_linear_params}
\end{table}

Considering the performances on the in-domain during training, we see in \autoref{tab:redux_linear_baselines_dev} reasonable baseline performances on the validation set. The validation scores described in \autoref{tab:redux_linear_baselines_dev} are not as strong as the state-of-the-art in-domain models \cite{Salminen:2020}, in fact they are comparable to the scores reported on the test set of the in the original paper \cite{Davidson:2017} which provided initial baseline scores.\footnote{We do not report these baseline scores as their work does not identify which weighting of their F1-score was used.}. While several previous work augment the textual data with syntactic knowledge \cite{Davidson:2017} or advanced token representations \cite{Salminen:2020} to boost classification performance, we only use the byte-pair encoded documents and the LIWC encoded documents, to ensure comparability with our experimental models. Moreover, as our primary concern is learning classifiers whose performance generalise to other datasets, unlike much prior work which concerns itself with learning classifiers that perform well within the dataset, we do not take further steps towards boosting our baseline classifiers in-domain performances.

\begin{table}[]
\centering
\resizebox{\textwidth}{!}{%
\begin{tabular}{l|clllll}
Training data             & Document Representation                     & Model                   & F1-macro & Accuracy & Precision & Recall   \\ \hline
\multirow{4}{*}{Davidson} & \multirow{2}{*}{BPE}                        & Logistic Regression     & $92.02$  & $95.52$  & $91.82$   & $92.22$  \\
                          &                                             & Support Vector Machine  & $92.13$  & $95.56$  & $91.73$   & $92.53$  \\
                          & \multirow{2}{*}{LIWC}                       & Logistic Regression     & $87.75$  & $93.09$  & $87.43$   & $88.08$  \\
                          &                                             & Support Vector Machine  & $89.02$  & $93.62$  & $87.64$   & $90.60$  \\
\multirow{4}{*}{Wulczyn}  & \multirow{2}{*}{BPE}                        & Logistic Regression     & $86.35$  & $95.67$  & $90.04$   & $83.42$  \\
                          &                                             & Support Vector Machine  & $86.47$  & $95.50$  & $89.02$   & $84.31$  \\
                          & \multirow{2}{*}{LIWC}                       & Logistic Regression     & $82.32$  & $94.86$  & $90.48$   & $77.34$  \\
                          &                                             & Support Vector Machine  & $82.98$  & $94.96$  & $90.14$   & $78.36$
\end{tabular}%
}
\caption{In-domain scores on validation set by linear baselines.}
\label{tab:redux_linear_baselines_dev}
\end{table}

\section{Experimental Models}

To evaluate which training dataset allows for better generalisation, we train our four models described in \autoref{sec:redux_neural} and their variations on each of our training dataset, resulting in $16$ different trained models. We show the best performing model parameters on the respective validation sets in \autoref{tab:redux_hyperparam_search}. In order to gain confidence intervals, we select a subset of these models and train them with $5$ different initial random seeds, to allow us to make claims of statistical significance of our models.

We train all of our neural network models following the same training procedure. We iterate over the training dataset in multiple epochs, shuffling the order of the data at the beginning of each epoch. As we train on a single task, the loss that is propagated through the network using backpropagation is computed on the validation set for the given task. To avoid over-training our model, we implement set our models to stop training after $15$ epochs of worse, that is strictly higher, loss values. As our training procedure closes, we apply the model on each test set, allowing us to evaluate its in-domain performance as well as its out-of-domain performance.

Using this training scheme, we define and search a hyper-parameter space for each model (see \autoref{sec:redux_neural} for the search space for each model and \autoref{tab:exp_model_parameters_davidson} and \autoref{tab:exp_model_parameters_wulczyn} for the best hyper-parameters for each model). Though some previous work \cite{Waseem:2018, CITE: Other papers that restrict vocabulary sizes} limit the vocabulary that is used to train models, we make no such limitations on the surface level tokens. Instead, for all models that use surface level representations, we pre-process the documents using the 200 dimensional Byte-Pair Encoding \cite{Heinzerling:2018} for two reasons: 1) computing the sub-words allows for a minimisation of the number of out-of-vocabulary tokens and 2) computing the sub-words also minimises the sizes of the vocabularies for each dataset. For all models that take documents represented through LIWC, we dramatically reduce the vocabulary to only the tokens that exist within LIWC, setting all other tokens to a token representing that it is out-of-vocabulary.

\zw{CODING: retrain the best parameters for the models with 4 new random seeds}


\zw{Add results and start by rough analysis of just numbers}
\zw{Look at predictions in detail, try to identify where they still fail}

\begin{landscape}
\begin{table}[]
\centering
\resizebox{1.4\textheight}{!}{%
\begin{tabular}{l|clllll}
  Training data             & Document Representation & Model                          & F1-macro & Accuracy & Precision & Recall \\ \hline
\mrow{8}{*}{\rot{Davidson}} & \mrow{4}{*}{\rot{BPE}}  & Multi-Layered Perceptron       &          &          &           &  \\
                            &                         & Convolutional Neural Network   &          &          &           &  \\
                            &                         & Long-Short Term Memory Network &          &          &           &  \\     
                            & \mrow{4}{*}{\rot{LIWC}} & Multi-Layered Perceptron       &          &          &           &  \\
                            &                         & Convolutional Neural Network   &          &          &           &  \\        
                            &                         & Long-Short Term Memory Network &          &          &           &  \\ \midrule       
\mrow{8}{*}{\rot{Wulczyn}}  & \mrow{4}{*}{\rot{BPE}}  & Multi-Layered Perceptron       &          &          &           &  \\        
                            &                         & Convolutional Neural Network   &          &          &           &  \\        
                            &                         & Long-Short Term Memory Network &          &          &           &  \\
                            & \mrow{4}{*}{\rot{LIWC}} & Multi-Layered Perceptron       &          &          &           &  \\
                            &                         & Convolutional Neural Network   &          &          &           &  \\        
                            &                         & Long-Short Term Memory Network &          &          &           &  
\end{tabular}%
}
\caption{In-domain scores on validation set by onehot encoded neural models.}
\label{tab:redux_onehot_neural_dev}
\end{table}
\end{landscape}


\begin{landscape}
\begin{table}[]
\centering
\resizebox{1.4\textheight}{!}{%
\begin{tabular}{l|clllll}
  Training data             & Document Representation & Model                          & F1-macro & Accuracy & Precision & Recall  \\ \hline
\mrow{8}{*}{\rot{Davidson}} & \mrow{4}{*}{\rot{BPE}}  & Multi-Layered Perceptron       &          &          &           & \\
                            &                         & Convolutional Neural Network   &          &          &           & \\
                            &                         & Long-Short Term Memory Network &          &          &           & \\
                            & \mrow{4}{*}{\rot{LIWC}} & Multi-Layered Perceptron       &          &          &           & \\
                            &                         & Convolutional Neural Network   &          &          &           & \\
                            &                         & Long-Short Term Memory Network &          &          &           & \\ \midrule
\mrow{8}{*}{\rot{Wulczyn}}  & \mrow{4}{*}{\rot{BPE}}  & Multi-Layered Perceptron       &          &          &           & \\
                            &                         & Convolutional Neural Network   &          &          &           & \\
                            &                         & Long-Short Term Memory Network &          &          &           & \\
                            & \mrow{4}{*}{\rot{LIWC}} & Multi-Layered Perceptron       &          &          &           & \\
                            &                         & Convolutional Neural Network   &          &          &           & \\
                            &                         & Long-Short Term Memory Network &          &          &           &
\end{tabular}%
}
\caption{In-domain scores on validation set by index encoded neural models.}
\label{tab:redux_index_neural_dev}
\end{table}
\end{landscape}

\section{Results}

\zw{UPDATE: Change F1, Accuracy, precision, recall names to full names from acc, prec, rec}
\begin{landscape}
\begin{table}[]
\centering
\resizebox{1.4\textheight}{!}{%
\begin{tabular}{ccl|llll|llll|llll|llll|llll}
                                     &                         &                                & \multicolumn{4}{c|}{Davidson} & \multicolumn{4}{c}{Wulczyn}   & \multicolumn{4}{c}{Waseem}    & \multicolumn{4}{c}{Waseem-Hovy} & \multicolumn{4}{c}{Garcia} \\
 Training data                       & Representation          & Model                          & F1    & acc    & prec & rec   & F1    & acc   & prec  & rec   & F1    & acc   & prec  & rec   & F1    & acc   & prec  & rec     & F1    & acc    & prec & rec   \\ \hline
\multirow{12}{*}{\rot{Davidson}}         & \mrow{6}{*}{\rot{BPE}}  & Support Vector Machine         &$92.19$&$95.60$&$91.91$&$92.47$&$71.26$&$86.83$&$67.87$&$92.47$&$49.06$&$68.64$&$49.56$&$49.41$&$58.49$&$65.66$&$49.41$&$58.25$  &$60.62$&$60.66$&$60.71$&$60.66$\\
                                     &                         & Logistic Regression            &$91.39$&$95.19$&$91.51$&$91.27$&$71.18$&$86.40$&$67.68$&$79.85$&$48.34$&$66.04$&$49.35$&$49.03$&$58.12$&$64.36$&$58.37$&$57.98$  &$58.15$&$58.15$&$58.16$&$58.15$\\
                                     &                         & Multi-Layered Perceptron       &$     $&$     $&$     $&$     $&$     $&$     $&$     $&$     $&$     $&$     $&$     $&$     $&$     $&$     $&$     $&$     $  &$     $&$     $&$     $&$     $\\
                                     &                         & Convolutional Neural Network   &$     $&$     $&$     $&$     $&$     $&$     $&$     $&$     $&$     $&$     $&$     $&$     $&$     $&$     $&$     $&$     $  &$     $&$     $&$     $&$     $\\
                                     &                         & Long-Short Term Memory network &$     $&$     $&$     $&$     $&$     $&$     $&$     $&$     $&$     $&$     $&$     $&$     $&$     $&$     $&$     $&$     $  &$     $&$     $&$     $&$     $\\
                                     & \mrow{6}{*}{\rot{LIWC}} & Support Vector Machine         &$86.85$&$92.01$&$84.41$&$90.12$&$75.32$&$90.02$&$72.41$&$79.78$&$53.29$&$71.24$&$53.27$&$54.42$&$55.89$&$65.36$&$57.59$&$55.93$  &$44.92$&$49.58$&$49.36$&$49.58$\\
                                     &                         & Logistic Regression            &$87.19$&$92.41$&$85.37$&$89.41$&$74.91$&$89.69$&$71.85$&$79.81$&$51.90$&$68.78$&$52.30$&$53.36$&$56.43$&$64.83$&$57.51$&$56.32$  &$45.23$&$48.74$&$48.31$&$48.74$\\
                                     &                         & Multi-Layered Perceptron       &$     $&$     $&$     $&$     $&$     $&$     $&$     $&$     $&$     $&$     $&$     $&$     $&$     $&$     $&$     $&$     $  &$     $&$     $&$     $&$     $\\
                                     &                         & Convolutional Neural Network   &$     $&$     $&$     $&$     $&$     $&$     $&$     $&$     $&$     $&$     $&$     $&$     $&$     $&$     $&$     $&$     $  &$     $&$     $&$     $&$     $\\
                                     &                         & Long-Short Term Memory network &$     $&$     $&$     $&$     $&$     $&$     $&$     $&$     $&$     $&$     $&$     $&$     $&$     $&$     $&$     $&$     $  &$     $&$     $&$     $&$     $\\\hline

\mrow{12}{*}{\rot{Wulczyn}}          & \mrow{6}{*}{\rot{BPE}}  & Support Vector Machine         &$63.89$&$70.55$&$66.24$&$78.47$&$86.42$&$95.59$&$89.11$&$84.15$&$53.45$&$80.34$&$55.30$&$53.23$&$51.95$&$67.90$&$59.86$&$54.19$  &$52.89$&$58.99$&$68.66$&$58.99$\\
                                     &                         & Logistic Regression            &$59.86$&$64.30$&$64.25$&$75.47$&$86.24$&$95.63$&$89.94$&$83.31$&$51.22$&$80.92$&$53.43$&$51.64$&$50.56$&$68.14$&$60.38$&$53.64$  &$50.04$&$57.53$&$68.78$&$57.53$\\
                                     &                         & Multi-Layered Perceptron       &$     $&$     $&$     $&$     $&$     $&$     $&$     $&$     $&$     $&$     $&$     $&$     $&$     $&$     $&$     $&$     $  &$     $&$     $&$     $&$     $\\
                                     &                         & Convolutional Neural Network   &$     $&$     $&$     $&$     $&$     $&$     $&$     $&$     $&$     $&$     $&$     $&$     $&$     $&$     $&$     $&$     $  &$     $&$     $&$     $&$     $\\
                                     &                         & Long-Short Term Memory network &$     $&$     $&$     $&$     $&$     $&$     $&$     $&$     $&$     $&$     $&$     $&$     $&$     $&$     $&$     $&$     $  &$     $&$     $&$     $&$     $\\
                                     & \mrow{6}{*}{\rot{LIWC}} & Support Vector Machine         &$76.06$&$81.72$&$73.73$&$88.34$&$83.20$&$95.05$&$90.72$&$78.42$&$54.49$&$82.51$&$59.19$&$54.12$&$49.82$&$68.02$&$59.77$&$53.23$  &$39.70$&$51.67$&$58.13$&$51.67$\\
                                     &                         & Logistic Regression            &$72.62$&$78.17$&$71.52$&$86.30$&$82.67$&$94.95$&$90.88$&$77.65$&$54.26$&$82.94$&$59.99$&$53.99$&$49.11$&$67.78$&$58.88$&$52.81$  &$39.41$&$51.67$&$58.77$&$51.67$\\
                                     &                         & Multi-Layered Perceptron       &$     $&$     $&$     $&$     $&$     $&$     $&$     $&$     $&$     $&$     $&$     $&$     $&$     $&$     $&$     $&$     $  &$     $&$     $&$     $&$     $\\
                                     &                         & Convolutional Neural Network   &$     $&$     $&$     $&$     $&$     $&$     $&$     $&$     $&$     $&$     $&$     $&$     $&$     $&$     $&$     $&$     $  &$     $&$     $&$     $&$     $\\
                                     &                         & Long-Short Term Memory network &$     $&$     $&$     $&$     $&$     $&$     $&$     $&$     $&$     $&$     $&$     $&$     $&$     $&$     $&$     $&$     $  &$     $&$     $&$     $&$     $
\end{tabular}%
}
\caption{Performance of linear baseline models and index-encoded models across in-domain and out-of-domain evaluation sets.}
\label{tab:redux_embedding_davidson}
\end{table}
\end{landscape}

\begin{landscape}
\begin{table}[]
\centering
\resizebox{1.4\textheight}{!}{%
\begin{tabular}{ccl|llll|llll|llll|llll|llll}
                                     &                         &                                 & \multicolumn{4}{c|}{Davidson} & \multicolumn{4}{c}{Wulczyn} & \multicolumn{4}{c}{Waseem} & \multicolumn{4}{c}{Waseem-Hovy} & \multicolumn{4}{c}{Garcia} \\
 Training data                       & Representation          & Model                           & F1 & acc & prec & rec         & F1 & acc & prec & rec       & F1 & acc & prec & rec      & F1 & acc & prec & rec           & F1 & acc & prec & rec      \\ \hline
\mrow{12}{*}{\rot{Davidson}}         & \mrow{6}{*}{\rot{BPE}}  & \textit{Support Vector Machine} &    &     &      &             &    &     &      &           &    &     &      &          &    &     &      &               &    &     &      &          \\
                                     &                         & \textit{Logistic Regression}    &    &     &      &             &    &     &      &           &    &     &      &          &    &     &      &               &    &     &      &          \\
                                     &                         & Multi-Layered Perceptron        &    &     &      &             &    &     &      &           &    &     &      &          &    &     &      &               &    &     &      &          \\
                                     &                         & Convolutional Neural Network    &    &     &      &             &    &     &      &           &    &     &      &          &    &     &      &               &    &     &      &          \\
                                     &                         & Long-Short Term Memory network  &    &     &      &             &    &     &      &           &    &     &      &          &    &     &      &               &    &     &      &          \\
                                     & \mrow{6}{*}{\rot{LIWC}} & \textit{Support Vector Machine} &    &     &      &             &    &     &      &           &    &     &      &          &    &     &      &               &    &     &      &          \\
                                     &                         & \textit{Logistic Regression}    &    &     &      &             &    &     &      &           &    &     &      &          &    &     &      &               &    &     &      &          \\
                                     &                         & Multi-Layered Perceptron        &    &     &      &             &    &     &      &           &    &     &      &          &    &     &      &               &    &     &      &          \\
                                     &                         & Convolutional Neural Network    &    &     &      &             &    &     &      &           &    &     &      &          &    &     &      &               &    &     &      &          \\
                                     &                         & Long-Short Term Memory network  &    &     &      &             &    &     &      &           &    &     &      &          &    &     &      &               &    &     &      &          \\\hline

\mrow{12}{*}{\rot{Wulczyn}}          & \mrow{6}{*}{\rot{BPE}}  & \textit{Support Vector Machine} &    &     &      &             &    &     &      &           &    &     &      &          &    &     &      &               &    &     &      &          \\
                                     &                         & \textit{Logistic Regression}    &    &     &      &             &    &     &      &           &    &     &      &          &    &     &      &               &    &     &      &          \\
                                     &                         & Multi-Layered Perceptron        &    &     &      &             &    &     &      &           &    &     &      &          &    &     &      &               &    &     &      &          \\
                                     &                         & Convolutional Neural Network    &    &     &      &             &    &     &      &           &    &     &      &          &    &     &      &               &    &     &      &          \\
                                     &                         & Long-Short Term Memory network  &    &     &      &             &    &     &      &           &    &     &      &          &    &     &      &               &    &     &      &          \\
                                     & \mrow{6}{*}{\rot{LIWC}} & \textit{Support Vector Machine} &    &     &      &             &    &     &      &           &    &     &      &          &    &     &      &               &    &     &      &          \\
                                     &                         & \textit{Logistic Regression}    &    &     &      &             &    &     &      &           &    &     &      &          &    &     &      &               &    &     &      &          \\
                                     &                         & Multi-Layered Perceptron        &    &     &      &             &    &     &      &           &    &     &      &          &    &     &      &               &    &     &      &          \\
                                     &                         & Convolutional Neural Network    &    &     &      &             &    &     &      &           &    &     &      &          &    &     &      &               &    &     &      &          \\
                                     &                         & Long-Short Term Memory network  &    &     &      &             &    &     &      &           &    &     &      &          &    &     &      &               &    &     &      &          \\
\end{tabular}%
}
\caption{Performance of onehot-encoded models across in-domain and out-of-domain datasets (\textit{italic} denotes baseline models).}
\label{tab:redux_embedding_davidson}
\end{table}
\end{landscape}
\zw{Add results and start by rough analysis of just numbers}
\zw{Add plots for development of loss over each epoch}
\zw{Add plots for F1 score during the evaluation set}
\zw{Look at predictions in detail, try to identify where they still fail}
\zw{Do logistic regression and to find out what clear patterns there are}

While the onehot and index encoded tensors should functionally be equal to one another, we see a direct influence of the input layers on the classification scores; with all models showing stronger performance using linear input layers. We propose that the reason for such discrepancies lie in the simpler training procedure of linear layers which don't seek to find relationships between different tokens but instead simply provide a linear function, and embedding layers that seek to identify the relationships between each all tokens in the dataset.

\section{Conclusions and future work}

\zw{Something about BERT based models}
BERT \cite{Koufakou,Vidgen,Tran:2020,Isaksen:2020}
As our aim is to consider the influence of LIWC-represented documents, we do not consider the more recent pre-trained Transformer-based language models \citep[e.g.]{Devlin:2019,Liu:2019} as the amounts of data necessary to train such a masked language model with LIWC representations are unavailable. Moreover, as the LIWC dictionary only occupies a small fraction of the entire English lexicon, and its tokens are abstractions on use of the language, training a language model is a fruitless endeavour.

\zw{Some concluding remarks}
While functionally this limits the vocabulary, there is also loss of information. Future work, could then employ both simple and complex mappings of different forms of words to single tokens that cohere with the LIWC dictionary, thus limiting information loss while retaining the predictive power.

\subsection{Limitations}
Although such lack of recognition can have positive effects, such as lower false positive rate, the politics of not being recognised, as argued by \citet{Benjamin:2019} are not straightforward and the lack of recognition does not provide a guarantee that systemic harm will not occur. For instance, if systems developed to detect abuse did not recognise Multi-cultural London English due to vocabulary reductions, any abuse that was written in that dialect would not be recognised, leaving those users in harms way. Given that LIWC was developed using ``dictionaries, thesauruses, questionnaires, and lists made by research assistants'' \citep{Tauscik:2010} in a North American context, it is highly unlikely that word forms that differ from mainstream usage were included. For instance, the commonly used `brotha' and `bruva' in North American and British contexts, respectively, are absent from the dictionary.
